\documentclass{article}
\usepackage{graphicx}
\usepackage[spanish]{babel}
\usepackage[utf8]{inputenc} % Acentos
\usepackage[noend]{algpseudocode}
\usepackage{algorithm}
\usepackage{tikz}
\usepackage{hyperref}
\usepackage[
top=3cm,
bottom=3cm,
left=2cm,
right=2cm,
heightrounded,
]{geometry}

\title{Proyecto 3}
\author{Alejandro Hernández Cano\\
    Martínez Santana Brayan\\
    Trad Mateos Kethrim Guadalupe}

\begin{document}
    \maketitle
    \section*{Repositorio: \url{https://github.com/BrayanCaro/BuscaminasVala}}

    \section{Introducción}

	Todos estamos familiarizados con el juego de buscaminas. El objetivo de este
	es simple: descubrir totalmente un campo minado sin llegar a activar una de
	estas y las reglas del juego sencillas. Además, puede ser un reto
	interesante al momento de programar, pues estar sujeto a varios diseños y
	depende de cada uno el que le parezca más adecuado.

	Este documento presenta y analiza la implementación de un Buscaminas en el
	lenguaje de programación Vala. Presentamos un proyecto completo con pruebas
	y documentación de un buen juego
	buscaminas que busca mejorar el que ya teníamos hecho al inicio de la
	carrera, haciendo un código más compacto, siguiendo principos antes
	extraños, que fueron vistos durante el curso y ofreciendo una interfaz al
	usuario más amigable que en el proyecto inicial.

	\section{Definición del problema}

	El reto de este proyecto consiste en implementar correctamente un juego de
	buscaminas en Vala, pero también se pueda guardar la partida en curso y
	poder cargar esta partida posteriormente. Esto se exitende desde aprender a
	utilizar el lenguaje Vala, manipular la serialización de objetos en este y
	poder realizar el diseño de el proyecto elegantemente.

	\section{Análisis del problema}

	Las reglas del buscaminas son sencillas. Inicia en un tablero de tamaño $n
	\times m$, con todas las casillas iguales. En cada turno el jugador
	decidirá presionar una de estas casillas o colocar una bandera. La bandera
	sirve únicamente para ayudarle al usuario a recordar las casillas en donde
	se cree hay una mina.

	Si el jugador decide presionar una casilla y esta contiene una mina, se
	perderá la partida. Si no contenía una mina entonces se extenderá la zona
	descubierta hasta que todas las casillas de la zona tengan al menos a una
	casilla con bomba adyacente a ella. En cada una de estas casillas se
	mostrará un número, que indica cuántas casillas adyacentes contienen una
	mina. Este número le servirá al usuario para poder deducir en cuales
	casillas hay minas y en cuales no las hay.

	Además, nuestro programa podrá ser capaz de:

    \begin{enumerate}
        \item Guardar y cargar partidas.
        \item Tener en una tabla los mejores puntajes.
        \item Dar la oportunidad de que el usuario decida el tamaño del tablero.
    \end{enumerate}

    \section{Mejor alternativa}

	Parte del reto de este proyecto fue el considerar en qué lenguaje realizar
	el juego. Después de varias opciones analizadas se terminó optando por, como
	ya fue mencionado anteriormente, Vala. A continuación daremos una pequeña
	descripción de este lenguaje y daremos varias justificaciones de que este
	lenguaje es buena propuesta para nuestro problema.

	Vala es un lenguaje de alto nivel orientado a objetos con un compilador
	que genera código en C. Vala es sintácticamente similar a C\# e incluye
	notables características como funciones anónimas, propiedades, genéricos,
	manejo de memoria asistida, manejo de excepciones, inferencia de tipos y el
	bucle foreach.

	Creemos que esta es una buena alternativa para realizar nuestro buscaminas
	pues es relativamente sencillo sintáctimanete y además ofrece varias
	características útiles, como las mencionadas anteriormente, que facilitan el
	escribir código. Además al ser compilado a C podemos esperar un programa
	final relativamente ligero y fácilmente distribuible.

    \section{Pseudocódigo}

	En esta sección describiremos los algoritmos utilizados para hacer posible
	este proyecto. Hay que tomar en cuenta que todos estos algoritmos están
	encapsulados dentro de una clase \texttt{Tablero}, por lo tanto hacemos la
	suposición de que al inicio de cada algoritmo, el tablero se encuentra en un
	estado válido, además, dentro de estos, denotaremos por $n$ y $m$ la
	cantidad de filas y columnas en el tablero, respectivamente y por $casillas$
	la matriz que representa las casillas del tablero.


	\begin{algorithm}
	\caption{Presionar casilla} \label{alg:presionar}
	\begin{algorithmic}[1]

	\Procedure{PresionarCasilla}{$x,y$}
	\Require \Call{CasillaValida}{$x,y$}
	\State $casillas[x,y]$.\Call{marcarPresionado}{ }
	\State \Call{Extender}{$x, y$}
	\EndProcedure

	\Procedure{Extender}{$x,y$}
	\If{$casillas[x,y].bombasVecinas > 0$}
	\Comment{Caso base}
	\State \Return
	\EndIf
	\State $coords \gets [ (x,y+1), (x-1,y), (x+1,y), (x,y-1) ]$
	\Comment{Vecinos arriba/abajo y de los lados}
	\ForAll{$(i, j) \in coords$}
	\If{\Call{CasillaValida}{$i,j$}}
	\State $c \gets casillas[i,j]$
	\If{\textbf{not} $c.minado$ \textbf{and not} $c.abanderado$ \textbf{and not}
	$c.presionado$}
	\State $c$.\Call{marcarPresionado}{ }
	\State \Call{Extender}{$i,j$}
	\EndIf
	\EndIf
	\EndFor
	\EndProcedure

	\Procedure{CasillaValida}{$x,y$}
	\If{$0 \leq x < n$ \textbf{and} $0 \leq y < m$}
	\State \Return \textbf{true}
	\EndIf
	\State \Return \textbf{false}
	\EndProcedure
	\end{algorithmic}
	\end{algorithm}


    \section{Pruebas y mejoras}    

	La mejora principal que vio nuestro programa es en el diseño. Como el
	buscaminas original fue realizado al inicio de la carrera, nuestros
	conocimientos en diseño eran muy pobres y como a penas estábamos practicando
	los conceptos básicos de programación orientada a objetos, termino estando
	muy mal diseñado y sin estar tan claro.

	Por otro lado, el nuevo buscaminas ve una mejora significativa en este
	rubro, reducimos la cantidad de clases notablemente simplificando el código,
	el diseño y la implementación, tanto en la interfaz con el usuario como en
	el modelo de nuestro tablero. Además esta vez incluimos pruebas unitarias,
	ayudándonos a depurar más facilemente nuestro código y poder tener más
	confianza en el correcto funcionamento.

	También, al utilizar serialización utilizando el formato json, es más fácil
	poder guardar estos tableros y leerlos por otros programas, si eso llegara
	a ser necesario.

	Otra gran mejora que se realizó fue la manera en la que se compila y ejecuta
	el programa. En las propuestas anteriores utilizabamos \texttt{make}, que a
	pesar de que cumplía con el objetivo, no había garantía de que lo hiciera en
	todos los ordenadores, pues nos enfrentamos con problema de compilación en
	algunos. Otra alternativa que se tenía era compilar todo el proyecto
	manualmente y tener que ejecutarlo utilizando el ejecutador de java. Esto
	resultaba algo engorroso y además se neciesitaba que el usuario tuviera
	conocimientos sobre java y tuviera y supiera utilizar las herramientas
	adecuadas.

	En esta implementación utilizamos \texttt{mason}, que resulta ser una
	alternativa más fiable que \texttt{make}, dando la misma facilidad para
	compilar utilizando un solo comando, pero teniendo más fiabilidad de que lo
	haga correctamente. Además esta herramienta arroja un archivio ejecutable,
	en vez de un archivo empaquetado \texttt{jar}, lo que facilita la
	distribución y ejecución de nuestro proyecto, sin siquiera necesitar
	instalar ninguna herramienta más.

    \section{Pensamiento a futuro}

    El programa podría ponerse en las tiendas de alguna plataforma de software como en 'Software de Ubuntu' porque Vala esta diseñado para la creación de aplicaciones (usando las bibliotecas de GNOME).

    \section*{Bibliografía}
    %
    % usar este comando para generar anexar links con
    % \biblio{Titulo_facil_de_entender}{link}
    %
    % Nota: Cuando copiamos y pegamos algunos links generan problemas con el guíon bajo '_', usar '\_' para no tener problemas
    %
    \newcommand*{\biblio}[2]{\item \href{#2}{#1}\\\textsf{#2}}
    \begin{enumerate}
        \biblio{Tutorial de Vala}{https://wiki.gnome.org/Projects/Vala/Tutorial/es}
        \biblio{Introducción a Vala}{https://www.youtube.com/watch?v=hkUnIy\_Viys}
    \end{enumerate}
\end{document}
